% Graduation Speech
% University of Colorado, Boulder
% Dept. of Computer Science
%
% Spring 2016
%
% Andy Sayler

\documentclass[11pt,twocolumn,letterpaper]{article}

% System Packages
\usepackage{epsfig}
\usepackage{float}
\usepackage[dvipsnames]{xcolor}
\usepackage{caption}
\usepackage{subcaption}
\usepackage{tabu}
\usepackage{color}
\usepackage{hyperref}
\usepackage{url}

% Local Packages
% None

% Package Options
\hypersetup{
    colorlinks,
    citecolor=black,
    filecolor=black,
    linkcolor=black,
    urlcolor=black
}

% Macros
\input{macros.tex}

% Other Options
\clubpenalty = 10000
\widowpenalty = 10000

% Start
\begin{document}

\title{Opportunities and Obligations for Computer
  Scientists\footnote{This speech was given during the Dept. of
    Computer Science Graduation ceremony at the University of
    Colorado, Boulder on May 6\textsuperscript{th}, 2016. Video of the
    speech is availble at \url{https://youtu.be/NElKYwHfehg}.}}

\author{Andy Sayler}

\date{May 6, 2016}

\maketitle

Thank you. As some of you may know, my name is Andy Sayler, or ``that
guy who did all the Operating Systems YouTube videos and who fixes the
broken virtual machines''. In any case, today I stand before you as a
fellow graduate of the Department of Computer Science. Now, I must
warn you, I wasn't given very much guidance regarding what to speak
about today. Which, I guess when you think about it, makes this a
pretty good analogy for post-graduate life: Allow me to be one of the
first to welcome you to the ``real world'' -- where we can no longer
rely on others to spell out, in detail, what they'd like for us to
do. Good luck!

I would also like to note that I am a bit of strange choice for
graduation speaker. After all, I'm like the Puxatony Phil of higher
education -- I came out of undergrad, saw my first monthly bills, and,
well, five more years of grad school. That makes this kind of like
asking Mick Jagger to give a speech about how to retire from Rock and
Roll... (Don't worry if you didn't get that one; it was for your
parents.)

But since I've been afforded the privilege of speaking to you today,
I'd like to commend everyone on their accomplishments over the last
four, five, or in the case of some of my fellow grad students, fifteen
years pursuing Computer Science degrees here at CU. Congratulations to
everyone gathered here today. The degrees you are being awarded are
certainly well earned, and will surely serve you well in the
future. You have joined the department during a time of unprecedented
growth and have excelled to yield the largest graduating class this
department has ever produced.

This is especially impressive given the youthfulness of Computer
Science as a field. We are likely unique across all of the graduation
ceremonies taking place today in that we have individuals in this room
who are older than our entire discipline. (And I note that not to
imply that anyone here is ``old'', but merely to point out that our
discipline is really very young.) Computer Science, as a field of
study, has only been around for the last sixty years or so; and our
own department hasn't even turned fifty. This fact presents numerous
opportunities that our peers in other fields will not be afforded.
Chief amongst those is the opportunity to interact with the founders
of our discipline. We are fortunate to have many such individuals
still alive, and in some cases, still active in our field. If I am to
impart a single suggestion to you today, it would be to strive to
exploit this fact. We are as physicists in the age of Newton, or
astronomers in the age of Galileo. Take the full advantage of this,
and endeavour to meet, work with, and learn from those that have paved
the road we now walk. It is a rare opportunity we have,and we will be
some of the last computer scientists ever afforded it.

So that's the first bit of potential wisdom I have for you. But here
is the second -- if ever presented with an opportunity to speak, at
length, to a captive audience, seize it. Doing so affords you one of
those rare forums where your audience is more or less compelled by
common politeness to listen to what you have to say. I'll also point
out that the challenge of accessing your smartphones inside these
robes only increases the novelty of this opportunity. And since I
have followed my own advice and accepted this invitation to speak,
permit me a moment of moralizing.

We are fortunate to work in a field that society, for better or worse,
values quite highly. And indeed, I sincerely hope that this valuation
brings you and your loved ones prosperous lives. But we must not take
such an investment for granted. The value society has placed on our
skills requires that we too seek to return value to society. Through
Computer Science we have the potential to provide great benefits to
the world. But we also run the risk of doing great harm.

It is not inconceivable that the work we do to increase efficiency
through the automation of day-to-day tasks will have a hand in the
elimination of large numbers of existing jobs, likely exacerbating the
growing issues of economic inequality our society already faces. The
work we do connecting the world via the Internet has facilitated
unprecedented levels of human communication, but it also raises
numerous privacy challenges as governments and corporations alike seek
to leverage these technologies to surveil the ways in which we speak.
Self-driving vehicles have promising environmental and safety
benefits, but these systems will also be required to make life and
death decisions dictated by the code we craft.

These are not challenges to be taken lightly, nor are they ones we can
afford to kick down the road for others to solve. These challenges
will not be solved by the talking heads on the evening news nor in the
comment threads of Internet. It is our duty as Computer Scientists,
and as humans, to ensure that we work to minimize and offset any harms
that may come from the goods we seek to provide.

We have all been afforded a great privilege to have been able to
obtain the degrees we now hold. We must apply that privilege in the
highest manner that we can. Having had the opportunity to know many
of you during our time here, I have no doubt that each and every one
of you can rise to meet these challenges head on and will work to
resolve them in a manner befitting of the degrees you now hold.

If you are able, volunteer some of your time to support the work of
worthy organizations who otherwise might not be able to afford your
talent. Endeavour to ensure others are afforded the same privileges we
have had, perhaps by teaching or mentoring those interested in our
work, or by donating to educational institutions such ours. And in
everything you do, be conscious of the ethical ramifications of our
work, and have the courage to speak up when you feel errors of
judgment are being made. Society values us and our skills very highly,
and we must endeavour to return the favor.

So with that parting thought, congratulations on your accomplishments,
best of luck in your future endeavours, and thank you for your time.

\end{document}
